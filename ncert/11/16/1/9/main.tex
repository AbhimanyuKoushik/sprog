\let\negmedspace\undefined
\let\negthickspace\undefined
\documentclass[journal]{IEEEtran}
\usepackage[a5paper, margin=10mm, onecolumn]{geometry}
%\usepackage{lmodern} % Ensure lmodern is loaded for pdflatex
\usepackage{tfrupee} % Include tfrupee package

\setlength{\headheight}{1cm} % Set the height of the header box
\setlength{\headsep}{0mm}     % Set the distance between the header box and the top of the text

\usepackage{gvv-book}
\usepackage{gvv}
\usepackage{cite}
\usepackage{amsmath,amssymb,amsfonts,amsthm}
\usepackage{algorithmic}
\usepackage{graphicx}
\usepackage{textcomp}
\usepackage{xcolor}
\usepackage{txfonts}
\usepackage{listings}
\usepackage{enumitem}
\usepackage{mathtools}
\usepackage{gensymb}
\usepackage{comment}
\usepackage[breaklinks=true]{hyperref}
\usepackage{tkz-euclide} 
\usepackage{listings}
% \usepackage{gvv}                                        
\def\inputGnumericTable{}                                 
\usepackage[latin1]{inputenc}                                
\usepackage{color}                                            
\usepackage{array}                                            
\usepackage{longtable}                                       
\usepackage{calc}                                             
\usepackage{multirow}                                         
\usepackage{hhline}                                           
\usepackage{ifthen}                                           
\usepackage{lscape}
\usepackage{multicol}

% Marks the beginning of the document
\begin{document}
\bibliographystyle{IEEEtran}
\vspace{3cm}

\title{11.16.1.9}
\author{EE24BTECH11018 - Durgi Swaraj Sharma}

% \maketitle
% \newpage
% \bigskip
{\let\newpage\relax\maketitle}
\renewcommand{\thefigure}{\theenumi}
\renewcommand{\thetable}{\theenumi}
\setlength{\intextsep}{10pt}
\numberwithin{equation}{enumi}
\numberwithin{figure}{enumi}
\renewcommand{\thetable}{\theenumi}
\textbf{Problem:} A box contains $1$ red and $3$ identical white balls. Two balls are drawn at random in succe
ssion without replacement. Write the sample space for this experiment.\\
\sbrak{\text{Additional: Find the probability mass function too.}}\\
\solution

The sample space $\sigma$ is given by:
\begin{align}
  \sigma = \cbrak{WW, RW, WR}
\end{align}

We shall define random variable $X$, which denotes the position at which the Red ball is drawn. \\
Then $X$ takes the values
\begin{align}
  X = 
  \begin{cases}
    0 ,&\, \text{red ball never drawn}\\
    1 ,&\, \text{red ball drawn first}\\
    2 ,&\, \text{red ball drawn second}
  \end{cases}
\end{align}
There is a total of $4\cdot3$ outcomes that can happen, with $4$ in first draw and $3$ in second.
These outcomes relate with $X$ as:
\begin{itemize}
  \item $X=0$ when both balls drawn are white, and there's a total of $3\cdot2$ these outcomes.
  \item $X=1$ when the first ball is red, the second ball has to white. Giving a total of $1\cdot3$.
  \item $X=2$ when the second ball is red, the first has to be white. Giving a total of $3\cdot1$
\end{itemize}
This translates to the Probability Mass Function as 
\begin{align}
  p_X\brak{k} = 
  \begin{cases}
    \frac{6}{12} = 0.50,\,& k = 0\\
    \frac{3}{12} = 0.25,\,& k = 1\\
    \frac{3}{12} = 0.25,\,& k = 2
  \end{cases}
\end{align}
\textbf{Computing the Probabilities}


We perform a Monte Carlo simulation to verify these probabilities by running a large number of trials. The algorithm proceeds as follows:
\begin{enumerate}
    \item Randomly select the first ball: Red with probability $\frac{1}{4}$, White with probability $\frac{3}{4}$.
    \item If the first ball is White, randomly select the second ball: Red with probability $\frac{1}{3}$, White with probability $\frac{2}{3}$.
    \item Count occurrences of each event $(WW, RW, WR)$.
    \item Compute empirical probabilities as the relative frequencies of these events.
\end{enumerate}
This is how the code works:
\begin{itemize}
    \item We generate $X_1$ by selecting a random number between 0 and 3. We set $X_1 = 1$ if the chosen number is 0, which occurs with probability $0.25$.
    \item If $X_1 = 1$, we set $X_2 = 0$ since the only red ball has already been drawn.
    \item Otherwise, we generate $X_2$ by selecting a random number between 0 and 2, setting $X_2 = 1$ if the chosen number is 0, which occurs with probability $0.33$.
\end{itemize}

\textbf{Outcome Definitions:}
\begin{align*}
    RW &: \quad X_1 = 1, X_2 = 0 \\
    WR &: \quad X_1 = 0, X_2 = 1 \\
    WW &: \quad X_1 = 0, X_2 = 0
\end{align*}


The estimated probabilities from the simulation are:
\begin{align}
  P(WW) &\approx 0.50 \\
  P(RW) &\approx 0.25 \\
  P(WR) &\approx 0.25
\end{align}

\end{document}
