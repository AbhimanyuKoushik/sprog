\let\negmedspace\undefined
\let\negthickspace\undefined
\documentclass[journal]{IEEEtran}
\usepackage[a5paper, margin=10mm, onecolumn]{geometry}
\usepackage{lmodern} % Ensure lmodern is loaded for pdflatex
\usepackage{tfrupee} % Include tfrupee package

\setlength{\headheight}{1cm} % Set the height of the header box
\setlength{\headsep}{0mm}     % Set the distance between the header box and the top of the text

\usepackage{gvv-book}
\usepackage{gvv}
\usepackage{cite}
\usepackage{amsmath,amssymb,amsfonts,amsthm}
\usepackage{algorithmic}
\usepackage{graphicx}
\usepackage{textcomp}
\usepackage{xcolor}
\usepackage{txfonts}
\usepackage{listings}
\usepackage{enumitem}
\usepackage{mathtools}
\usepackage{gensymb}
\usepackage{comment}
\usepackage[breaklinks=true]{hyperref}
\usepackage{tkz-euclide} 
\usepackage{listings}
\def\inputGnumericTable{}                                 
\usepackage[latin1]{inputenc}                                
\usepackage{color}                                            
\usepackage{array}                                            
\usepackage{longtable}                                       
\usepackage{calc}                                             
\usepackage{multirow}                                         
\usepackage{hhline}                                           
\usepackage{ifthen}                                           
\usepackage{lscape}

\begin{document}

\bibliographystyle{IEEEtran}
\vspace{3cm}

\title{11.16.3.21.2}
\author{EE24TECH11036 - Krishna Patil}
% \maketitle
% \newpage
% \bigskip
{\let\newpage\relax\maketitle}
\renewcommand{\thefigure}{\theenumi}
\renewcommand{\thetable}{\theenumi}
\setlength{\intextsep}{10pt} % Space between text and floats


\textbf{Question}: In a class of $60$ students, $30$ opted for NCC, $32$ opted for NSS and $24$ opted for both NCC and NSS. If one of these students is selected at random, find the probability that the student has opted neither NCC nor NSS. \\ \\
\solution
Define events $X$ and $Y$ as shown in the table \ref{table}, \\
\begin{table}[h!]    
  \centering
   \begin{table}[h!]
    \centering
    \begin{tabular}{|c|c|}
        \hline
        \textbf{One end of Jumper Wire}  & \textbf{Another end of Jumper Wire} \\
        \hline
          Digital pin 0 & Push button 16 \\
          Digital pin 1 & Push button 17 \\
          Digital pin 2 & LCD pin 4 \\
          Digital pin 3 & LCD pin 6 \\
          Digital pin 4 & LCD pin 11 \\
          Digital pin 5 & LCD pin 12 \\
          Digital pin 6 & LCD pin 13 \\
          Digital pin 7 & LCD pin 14 \\
          Digital pin 8 & Push button 18 \\
          Digital pin 9 & Push button 19 \\
          Digital pin 10 & Push button 20 \\
          Digital pin 11 & Push button 21 \\
          Digital pin 12 & Push button 22 \\
          Digital pin 13 & Push button 23 \\
          Analog pin A1 & Push button 15 \\
          Analog pin A2 & Push button 14 \\
          Analog pin A3 & Push button 13 \\
          Analog pin A4 & Push button 12 \\
          Analog pin A5 & Push button 11 \\
          Analog pin A0 & Push buttons 1-10 (digit buttons)\\
          LCD pin 1 & Ground \\
          LCD pin 2 & 5V \\
          LCD pin 15 & 5V via 1k $\Omega$ resistor \\
          LCD pin 16 & Ground \\
          LCD pin 3 & Ground via 1.5 k $\Omega$ resistor \\
          LCD pin 5 & Ground \\
          LCD pin 5 & All push buttons \\
          
        \hline
    \end{tabular}
\end{table}

  \caption{defining events}
  \label{table}
\end{table}
\newline Below are some posulates and theorems from boolean algebra : 
\begin{table}[h!]    
  \centering
   \begin{table}[h!]
    \centering
    \begin{tabular}{|c|c|}
        \hline
        \textbf{Button number}  & \textbf{Function} \\
        \hline
           1 - 10 & Digits 0 - 9 \\
           11 & Clear \\
           12 & $\ln{(x)}$ and $\log{(x)}$ \\
           13 & Right Parenthesis \\
           14 & $\sin{(x)}$, $\cos{(x)}$, and $\tan{(x)}$ \\
           15 & $e$ and $\pi$ \\
           16 & Backspace \\
           17 & Decimal Point \\
           18 & Equal To \\
           19 & Left Parenthesis \\
           20 & Division (/)\\
           21 & Multiplications (*)\\
           22 & Subtraction(-) \\
           23 & Addition (+) \\
        \hline
    \end{tabular}
\end{table}

  \caption{Boolean Algebra}
  \label{table2}
\end{table}
\newline The axioms of probability are as follows:

\textbf{Non-Negativity Axiom:}
\[
P(A) \geq 0
\]
The probability of any event \( A \) is always non-negative.

\textbf{Normalization Axiom:}
\[
P(S) = 1
\]
The probability of the sample space \( S \) (i.e., the set of all possible outcomes) is 1.

\textbf{Additivity Axiom (Countable Additivity for Disjoint Events):}  
If \( A_1, A_2, A_3, \dots \) are mutually exclusive (disjoint) events, then:
\[
P(A_1 \cup A_2 \cup A_3 \cup \dots) = P(A_1) + P(A_2) + P(A_3) + \dots
\]

 For any two event A and B,
\begin{align}
	\because A + A^\prime &= 1 \\
	 AB + A^\prime B &= B \label{2} \\
	 \implies \pr{AB} + \pr{A^\prime B} &= \pr{B} \label{3} \\
	 \because B + B^\prime &= 1 \\
	 AB + AB^\prime &= A \label{5}\\
	 \implies \pr{AB} + \pr{AB^\prime} &= \pr{A} \label{6} \\
	 \text{adding } \eqref{2} \text{ and } \eqref{5} \\
	 A + B &= AB + AB + AB^\prime + A^\prime B  \\
	 A + B &= AB + AB^\prime + A^\prime B \\ 
	 \pr{A + B} &= \pr{AB} + \pr{AB^\prime} + \pr{A^\prime B} \label{10}\\
	 \text{Adding \eqref{3},\eqref{6} and \eqref{10} and cancelling same terms } \\
	 \pr{AB} &= \pr{A} + \pr{B} - \pr{A + B} \\
	 \because \pr{A^\prime B^\prime} &=  \pr{\brak{A + B}^\prime} \label{13} \\
	 \pr{A^\prime  B^\prime} &=  1 - \pr{A+B} \label{14}
\end{align}



From the given data in question,
    \begin{align}
        \pr{A} &= \frac{30}{60} \\
        \pr{B} &= \frac{32}{60} \\
        \pr{AB} &= \frac{24}{60}     
    \end{align}
Now using axioms of probability (boolean logic),
Thus, we write
    \begin{align}
	    \pr{A + B} &= \pr{A} +  \pr{B} - \pr{A B} \\
	                                 &= \frac{30}{60} + \frac{32}{60} - \frac{24}{60} \\
	                                 &= \frac{38}{60} \\
	     \pr{A^\prime  B^\prime} &=  1 - \pr{A+B} \brak{\because \eqref{13} \text{ and } \eqref{14}} \\ 
	                              &= 1- \frac{38}{60} = \frac{11}{30} 
    \end{align} 
So, the probablity $\pr{A^\prime  B^\prime}$ i.e., the probability that the student has opted neither NCC nor NSS is $\frac{11}{30} = 0.36667$.
Also after verifying using computational method to get the probability as 0.36680.
\end{document}


