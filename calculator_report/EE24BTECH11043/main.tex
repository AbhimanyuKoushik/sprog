\let\negmedspace\undefined
\let\negthickspace\undefined
\documentclass[journal,12pt,onecolumn,article]{IEEEtran}
\usepackage{cite}
\usepackage{color,soul}
\usepackage{amsmath,amssymb,amsfonts,amsthm}
\usepackage{algorithmic}
\usepackage{float}
\usepackage{graphicx}
\usepackage{textcomp}
\usepackage{xcolor}
\usepackage{txfonts}
\usepackage{listings}
\usepackage{enumitem}
\usepackage{mathtools}
\usepackage{gensymb}
\usepackage{comment}
\usepackage[breaklinks=true]{hyperref}
\usepackage{multicol}
\usepackage{longtable}
\usepackage{multirow}
\usepackage{hhline}
\usepackage{ifthen}
\usepackage{lscape}
\usepackage[latin1]{inputenc} 

\newtheorem{theorem}{Theorem}[section]
\newtheorem{problem}{Problem}
\newtheorem{proposition}{Proposition}[section]
\newtheorem{lemma}{Lemma}[section]
\newtheorem{corollary}[theorem]{Corollary}
\newtheorem{example}{Example}[section]
\newtheorem{definition}[problem]{Definition}

\newcommand{\BEQA}{\begin{eqnarray}}
\newcommand{\EEQA}{\end{eqnarray}}
\newcommand{\define}{\stackrel{\triangle}{=}}
\renewcommand{\thesection}{\arabic{section}}

\usepackage{titlesec}
\titleformat{\section}{\raggedright\normalfont\Large}{\thesection.}{1em}{}
\titleformat{\subsection}{\raggedright\normalfont\large}{\thesubsection.}{1em}{}

\theoremstyle{remark}
\newtheorem{rem}{Remark}

\begin{document}

\title{Hardware Experiment of Scientific Calculator}
\author{Murra Rajesh Kumar Reddy - EE24BTECH11043}
\maketitle
\section{Introduction}
This project involves designing a calculator using an Arduino Uno, a 16x2 LCD display, a 4x4 keypad, and a shift button to toggle between basic arithmetic and scientific operations. The objective is to create a user-friendly calculator with an interactive interface.

\section{Components Used}
The hardware components used in this project include:
\begin{enumerate}
    \item \textbf{Arduino Uno} - Microcontroller for processing inputs and performing calculations.
    \item \textbf{16x2 LCD Display} - Displays user inputs and computed results.
    \item \textbf{4x4 Keypad} - Allows user input for numbers and mathematical operations.
    \item \textbf{Push Button} - Used to shift between basic and scientific functions.
    \item \textbf{Resistors and Wires} - Supporting electronic components.
\end{enumerate}

\section{Working Principle}
\subsection{User Input}
The user enters numbers and operations via the 4x4 keypad.

\subsection{Processing}
The Arduino reads input values and processes calculations accordingly.

\subsection{Shift Functionality}
A button is used to toggle between basic arithmetic functions (+, -, $\times$, /) and scientific functions (sin, cos, tan, log, etc.).

\subsection{Output Display}
The computed result is displayed on the LCD screen.

\section{Features}
\begin{enumerate}
    \item Performs basic arithmetic calculations: addition, subtraction, multiplication, and division.
    \item Supports scientific functions such as trigonometric and logarithmic calculations.
    \item User-friendly interface with a clear LCD display.
    \item The shift button enables additional functionalities without extra hardware.
\end{enumerate}

\section{Circuit Pin Connections}
\centering
\begin{tabular}{|l|l|l|l|}
\hline
\textbf{LCD Pin} & \textbf{Function} & \textbf{Arduino Pin} & \textbf{Notes} \\ \hline
RS             & Register Select & D8 (PB0)           & LOW: Command; HIGH: Data \\ \hline
E              & Enable          & D9 (PB1)           & Latches data \\ \hline
D4             & Data Bit 4      & D10 (PB2)          & 4-bit interface \\ \hline
D5             & Data Bit 5      & D11 (PB3)          & 4-bit interface \\ \hline
D6             & Data Bit 6      & D12 (PB4)          & 4-bit interface \\ \hline
D7             & Data Bit 7      & D13 (PB5)          & 4-bit interface \\ \hline
\end{tabular}

\bigskip
Additional connections: VSS (GND), VDD (+5V), VO (potentiometer for contrast), RW (GND), A (+5V backlight), K (GND).

\subsubsection{Push Button Connections}
\centering
\begin{tabular}{|l|l|l|l|}
\hline
\textbf{Button}  & \textbf{Function}         & \textbf{Arduino Pin} & \textbf{Notes} \\ \hline
Digits 0--5    & Enter digits 0--5         & D2, D3, D4, D5, D6, D7   & Active-low; internal pull-ups \\ \hline
Digits 6--9    & Enter digits 6--9         & A0, A1, A2, A3           & Digital inputs; pull-ups enabled \\ \hline
\textbf{Clear (C)}    & Clear input                & A4                      & Resets expression \\ \hline
\textbf{Enter (=)}   & Evaluate expression        & A5                      & Triggers computation \\ \hline
Extra Button 1 & Additional function       & (Assign as needed)      & \\ \hline
Extra Button 2 & Additional function       & (Assign as needed)      & \\ \hline
\end{tabular}

\bigskip
\textbf{Note:} The extra two push buttons can be used for further functions or for shift modes if needed.

\section{Code Structure}
The code follows a modular approach, with key sections handling different functionalities:

\subsection{Library Inclusions}
The necessary libraries are included at the beginning to interface with the LCD and perform mathematical operations.
\begin{lstlisting}[language=C++, caption=Library Inclusion]
#include <LiquidCrystal.h>  // LCD display control
#include <math.h>           // Mathematical functions (sin, cos, etc.)
\end{lstlisting}

\subsection{Hardware Setup}
The LCD display is connected to the Arduino through digital pins, and the push buttons are assigned to specific pins for input handling. A potentiometer is used for selecting different functions.
\begin{lstlisting}[language=C++, caption=Hardware Setup]
LiquidCrystal lcd(7, 8, 9, 10, 11, 12);  // LCD pins
const int potPin = A0;  // Potentiometer for function selection
const int buttonPins[] = {2, 3, 4, 5, 6}; // Buttons for input
\end{lstlisting}

\subsection{Input Handling}
The push buttons allow users to enter numbers and select operations. The potentiometer maps to different scientific functions.
\begin{lstlisting}[language=C++, caption=Reading Button Input]
void readButtons() {
    for (int i = 0; i < 5; i++) {
        if (digitalRead(buttonPins[i]) == LOW) {
            // Capture input value
        }
    }
}
\end{lstlisting}

\begin{lstlisting}[language=C++, caption=Mapping Potentiometer Input]
int getFunction() {
    int value = analogRead(potPin);
    return map(value, 0, 1023, 0, 5);  // Example: 0 = sin, 1 = cos, etc.
}
\end{lstlisting}

\subsection{Performing Calculations}
A function processes user input and performs the required mathematical operation.
\begin{lstlisting}[language=C++, caption=Calculation Function]
float calculate(float num1, float num2, char op) {
    switch (op) {
        case '+': return num1 + num2;
        case '-': return num1 - num2;
        case '*': return num1 * num2;
        case '/': return (num2 != 0) ? num1 / num2 : 0;
        case 's': return sin(num1);  // Example: Scientific function
        default: return 0;
    }
}
\end{lstlisting}

\subsection{Displaying Output}
Results are displayed on the \textbf{16x2 LCD screen} after processing.
\begin{lstlisting}[language=C++, caption=LCD Output Display]
void displayResult(float result) {
    lcd.clear();
    lcd.setCursor(0, 0);
    lcd.print("Result:");
    lcd.setCursor(0, 1);
    lcd.print(result);
}
\end{lstlisting}

\subsection{Main Loop Execution}
The loop continuously checks for user input, processes calculations, and updates the LCD display.
\begin{lstlisting}[language=C++, caption=Main Loop Execution]
void loop() {
    int selectedFunction = getFunction();
    readButtons();
    // Perform calculation and display result
}
\end{lstlisting}

\section{Features of the Code}
\begin{enumerate}
    \item Supports \textbf{basic arithmetic} operations (+, -, $\times$, /)
    \item Includes \textbf{scientific functions} (sin, cos, log, etc.)
    \item Uses a \textbf{potentiometer} for function selection
    \item Displays results on an \textbf{LCD screen}
    \item Handles \textbf{multiple inputs} via push buttons
\end{enumerate}
\section{Challenges and Solutions}
\begin{enumerate}
    \item \textbf{Keypad Debouncing:} Implemented software techniques to handle multiple keypresses correctly.
    \item \textbf{Limited LCD Display Space:} Optimized data display by clearing and refreshing necessary portions.
    \item \textbf{Memory Constraints:} Efficiently managed code structure to ensure smooth operation.
\end{enumerate}

\section{Conclusion}
This project successfully demonstrates the implementation of a calculator using an Arduino. It serves as an educational tool for understanding microcontroller-based projects and user interface design.

\end{document}

