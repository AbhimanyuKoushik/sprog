\documentclass[journal]{IEEEtran}
\usepackage[a5paper, margin=10mm, onecolumn]{geometry}
\usepackage{tfrupee} % Include tfrupee package

\setlength{\headheight}{1cm} % Set the height of the header box
\setlength{\headsep}{0mm}     % Set the distance between the header box and the top of the text

\usepackage{cite}
\usepackage{amsmath,amssymb,amsfonts,amsthm}
\usepackage{algorithmic}
\usepackage{graphicx}
\usepackage{textcomp}
\usepackage{xcolor}
\usepackage{txfonts}
\usepackage{listings}
\usepackage{enumitem}
\usepackage{mathtools}
\usepackage{gensymb}
\usepackage{comment}
\usepackage[breaklinks=true]{hyperref}
\usepackage{tkz-euclide}
\usepackage{multicol}
\newcommand{\brak}[1]{\left( #1 \right)}
\renewcommand{\thetable}{\arabic{table}}

% Marks the beginning of the document
\begin{document}
\bibliographystyle{IEEEtran}
\vspace{3cm}
\title{Calculator}
\author{EE24BTECH11017 - Karthik}
{\let\newpage\relax\maketitle}
\renewcommand{\thefigure}{\theenumi}
\renewcommand{\thetable}{\theenumi}

\section{Introduction}
This report presents the design and implementation of a scientific calculator using an Arduino Uno, a JHD162A LCD display, 16 input buttons, and other supporting components. The calculator is programmed using AVR-GCC on Termux (Debian) and supports various mathematical functions within the constraints of 16 push buttons. The report covers the circuit design, software implementation, and testing results.



\section{Hardware Components}
The main components used in this project are listed in Table 
\begin{table}[h!]
\centering
\begin{tabular}{|c|c|}
\hline
\textbf{Component} & \textbf{Quantity} \\
\hline
Arduino Microcontroller & 1 \\
\hline
Non-I2C 16x2 LCD & 1 \\
\hline
Push Buttons (0-9 digits) & 10 \\
\hline
Push Buttons (+, -, *, /) & 4 \\
\hline
Push Button (Scroll for sin, cos, tan) & 1 \\
\hline
15k\textohm Resistors & 10 \\
\hline
1k\textohm Resistors & 1 \\
\hline
Breadboard & 1 \\
\hline
Jumper Wires & As required \\
\hline
\end{tabular}
\caption{List of Components}
\label{tab:components}
\end{table}

\section{Circuit Connections and Keypad Interface}
To optimize space, the circuit connections and keypad functions are presented in Tables below respectively.

\begin{table}[h!]
\centering
\small
\begin{tabular}{|c|c|}
\hline
\textbf{Component} & \textbf{Arduino Pin} \\
\hline
LCD RS & D7 \\
\hline
LCD E & D6 \\
\hline
LCD D4 & D5 \\
\hline
LCD D5 & D4 \\
\hline
LCD D6 & D3 \\
\hline
LCD D7 & D2 \\
\hline
Keypad Rows & A0-A3 \\
\hline
Keypad Columns & D8-D11 \\
\hline
Power (VCC) & 5V \\
\hline
Ground (GND) & GND \\
\hline
\end{tabular}
\caption{Circuit Connections}
\label{tab:connections}
\end{table}

\vspace{-3mm}

\begin{table}[h!]
\centering
\small
\begin{tabular}{|c|c|}
\hline
\textbf{Button} & \textbf{Primary / Alternate Mode Function} \\
\hline
0-9 & Numeric Input \\
\hline
+ & Addition \\
\hline
- & Subtraction \\
\hline
$\times$ & Multiplication \\
\hline
$\div$ & Division \\
\hline
= & Compute Result \\
\hline
C & Clear Input \\
\hline
. & Decimal Point \\
\hline
M+ & Store in Memory \\
\hline
MR & Recall Memory \\
\hline
MC & Clear Memory \\
\hline
SIN & $\sin x$ / $\sin^{-1} x$ \\
\hline
COS & $\cos x$ / $\cos^{-1} x$ \\
\hline
TAN & $\tan x$ / $\tan^{-1} x$ \\
\hline
EXP & $e^x$ / $\ln x$ \\
\hline
LOG & $\log_{10} x$ / $\log_2 x$ \\
\hline
MODE & Toggle Between Standard and Scientific Mode \\
\hline
\end{tabular}
\caption{Keypad Button Functions}
\label{tab:keypad}
\end{table}


\section{Software Implementation}
The firmware is written in Embedded C and compiled using AVR-GCC. The core functionalities include:
\begin{itemize}
\item Reading ADC values for digit buttons.
\item Detecting arithmetic operations through digital pins.
\item Implementing additional functions such as memory storage and recall.
\item Displaying input and results dynamically on the LCD.
\item Handling error conditions such as division by zero.
\end{itemize}

\section{Features}
\begin{itemize}
    \item \textbf{Basic Arithmetic}: Addition, subtraction, multiplication, and division.
    \item \textbf{Scientific Functions}: Trigonometry, logarithms, exponentiation.
    \item \textbf{Memory Functions}: Store and recall values.
    \item \textbf{Number Systems}: Convert between HEX, DEC, BIN, and OCT.
    \item \textbf{Multi-Mode Input}: Normal, Alpha, and Shift functions.
    \item \textbf{Parentheses Handling}: Supports complex expressions.
\end{itemize}

\section{Conclusion}
This project successfully implements a digital calculator using an Arduino microcontroller. By leveraging ADC values and optimizing pin usage, the design maintains hardware simplicity while delivering comprehensive functionality. The use of AVR-GCC enhances efficiency, and future improvements could include higher precision and additional functions.

\end{document}

