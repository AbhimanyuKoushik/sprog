\let\negmedspace\undefined
\let\negthickspace\undefined
\documentclass[journal]{IEEEtran}
\usepackage[a5paper, margin=10mm, onecolumn]{geometry}

\setlength{\headheight}{1cm} % Set header height
\setlength{\headsep}{0mm}     % Set space between header and text

\usepackage{cite}
\usepackage{amsmath,amssymb,amsfonts,amsthm}
\usepackage{algorithmic}
\usepackage{graphicx}
\usepackage{textcomp}
\usepackage{xcolor}
\usepackage{listings}
\usepackage{enumitem}
\usepackage{mathtools}
\usepackage{gensymb}
\usepackage{hyperref}
\usepackage{longtable}
\usepackage{multirow}
\usepackage{hhline}
\usepackage{lscape}

\renewcommand{\thefigure}{\theenumi}
\renewcommand{\thetable}{\theenumi}
\setlength{\intextsep}{10pt} % Space between text and figures

\numberwithin{equation}{enumi}
\numberwithin{figure}{enumi}
\renewcommand{\thetable}{\theenumi}

% Document start
\begin{document}
\bibliographystyle{IEEEtran}

\title{Implementation of a Digital Time Display System}
\author{EE24BTECH11008-Aslin Garvasis}

{\let\newpage\relax\maketitle}

\section*{Overview}
This project focuses on the development of a digital time display system using an ATmega328P microcontroller. The system supports three functionalities: clock, stopwatch, and countdown timer. Time management is handled via Timer1 interrupts, and user input is managed through push-button switches. The output is displayed on a multiplexed seven-segment display.

\section*{Components Used}
The project utilizes the following hardware components:
\begin{itemize}
    \item ATmega328P microcontroller
    \item Six seven-segment common-anode displays
    \item Three push buttons for mode selection
    \item Resistors and interconnecting wires
\end{itemize}

\section*{Software Implementation}
The firmware is developed in C using AVR-GCC libraries, following a structured programming approach:

\subsection{System Initialization}
The setup function configures the microcontroller’s GPIO pins and initializes Timer1 to generate periodic interrupts for timekeeping. Additionally, internal pull-up resistors are enabled for the buttons.

\subsection{Timer Interrupt Handling}
A dedicated Timer1 interrupt service routine (ISR) executes every second to update time values across different modes and handle day transitions where necessary.

\subsection{Mode Control}
User input is processed through three buttons, each corresponding to a specific functionality:
\begin{itemize}
    \item Button 1: Activates standard clock mode
    \item Button 2: Starts or resets the stopwatch mode
    \item Button 3: Initiates or resets the countdown timer mode
\end{itemize}

\subsection{Multiplexing of Seven-Segment Display}
To efficiently utilize microcontroller pins, display multiplexing is implemented, allowing one digit to be displayed at a time while cycling through the digits at high speed to create a continuous display effect.

\section*{Results and Analysis}
The system was successfully built and tested. It accurately updates and displays time across all three operational modes. The multiplexing technique ensures efficient power consumption and GPIO utilization.

\section*{Potential Enhancements}
Future improvements could include:
\begin{itemize}
    \item Integration of an RTC (Real-Time Clock) module for enhanced accuracy
    \item Adding an alarm function
    \item Implementing an OLED display for better readability and additional features
\end{itemize}

\section*{Conclusion}
This project successfully demonstrates the implementation of a digital clock, stopwatch, and timer using an ATmega328P microcontroller. By leveraging Timer1 interrupts, the system achieves precise timekeeping, and multiplexing minimizes hardware complexity.

\section*{References}
This assignment was developed with reference to open-source AVR-GCC documentation and discussions with peers, including contributions from Dhawal (EE24BTECH11015).

\end{document}


