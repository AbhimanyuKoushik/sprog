\documentclass{article}
\usepackage{graphicx}
\usepackage{amsmath}
\usepackage{hyperref}

\title{\textbf{DIGITAL CLOCK}}
\author{EE24BTECH11038 - M.B.S Aravind}
\date{March 24, 2025}

\begin{document}

\maketitle

\tableofcontents

\section{Required Components}
\begin{itemize}
    \item Breadboard
    \item Arduino UNO (or ATmega328P Microcontroller)
    \item Jumper Cables
    \item 6x Seven-Segment Displays
    \item 7447 BCD Decoder
    \item Resistors
    \item Push Buttons
\end{itemize}

\section{Hardware Connections}
\begin{table}[h]
    \centering
    \begin{tabular}{|c|c|l|}
        \hline
        \textbf{Component} & \textbf{ATmega328P Pin} & \textbf{Connection Description} \\
        \hline
        BCD Input A & Digital Pin 2 & Connected to 7447 A input \\
        BCD Input B & Digital Pin 3 & Connected to 7447 B input \\
        BCD Input C & Digital Pin 4 & Connected to 7447 C input \\
        BCD Input D & Digital Pin 5 & Connected to 7447 D input \\
        Common Anode Pins & PORTC Analog Pins & Control individual display digits \\
        Mode Button & PB0 & Switch between Clock, Timer, Stopwatch \\
        Start/Stop Button & PB1 & Control mode-specific functions \\
        \hline
    \end{tabular}
    \caption{Hardware Connections}
\end{table}

\section{Working Explanation}
\subsection{Initialization of I/O and Timer}
The AVR microcontroller initializes critical system components:
\begin{itemize}
    \item Configures BCD output pins for 7-segment display control.
    \item Sets up Timer1 for interrupt-driven time updates.
    \item Enables pull-up resistors for button inputs.
    \item Initializes global interrupt mechanism.
\end{itemize}

\subsection{Displaying Time Using Multiplexing}
The clock implements efficient display rendering:
\begin{itemize}
    \item Extracts individual digits for hours, minutes, and seconds.
    \item Uses Binary-Coded Decimal (BCD) encoding.
    \item Activates one display digit at a time.
    \item Rapidly switches between digits to create a persistent vision effect.
    \item Minimizes I/O pin usage through sequential activation.
\end{itemize}

\subsection{Time Keeping and Increment Logic}
Time management follows precise rules:
\begin{itemize}
    \item Seconds increment every interrupt cycle.
    \item Automatic rollover for seconds (60 $\rightarrow$ 00).
    \item Minute increment when seconds reach 60.
    \item Hour increment when minutes reach 60.
    \item 24-hour cycle completion with hour reset.
\end{itemize}

\subsection{Timer1 Interrupt for Precise Timing}
Implemented using Clear Timer on Compare Match (CTC) Mode:
\begin{itemize}
    \item 1-second interrupt generation.
    \item Precise time tracking independent of the main loop.
    \item Automatic time progression.
    \item Modulo arithmetic for time rollover.
\end{itemize}

\subsection{Main Loop Execution}
The main function provides:
\begin{itemize}
    \item Continuous display refresh.
    \item Non-blocking operation.
    \item Smooth time update mechanism.
    \item Potential for future feature expansion.
\end{itemize}

\section{Code Architecture}
\subsection{Key Design Characteristics}
\begin{itemize}
    \item Interrupt-driven time management.
    \item Efficient memory utilization.
    \item Modular mode switching.
    \item Debounced button handling.
\end{itemize}

\subsection{Interrupt Service Routine (ISR) Features}
\begin{itemize}
    \item Precise 1-second time incrementation.
    \item Cascading time update logic.
    \item Automatic carry propagation.
    \item Mode-specific time tracking.
\end{itemize}

\section{Conclusion}
The implemented digital clock demonstrates:
\begin{itemize}
    \item Efficient microcontroller-based timekeeping.
    \item Flexible multi-mode functionality.
    \item Robust interrupt-driven design.
    \item Scalable embedded system architecture.
\end{itemize}

Potential future improvements:
\begin{itemize}
    \item Real-Time Clock (RTC) module integration.
    \item Battery backup implementation.
    \item Enhanced user interface.
    \item Additional mode functionalities.
\end{itemize}

\section{References}
\begin{itemize}
    \item Code by rongali charan
    \item AI suggestions
\end{itemize}

\end{document}
  
