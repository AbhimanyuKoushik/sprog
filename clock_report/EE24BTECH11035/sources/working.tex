

\subsection{Initialization of I/O and Timer}
The ATmega328P microcontroller initializes the required I/O pins and configures Timer1 for interrupt-driven time updates.

\begin{itemize}
    \item The BCD pins (A, B, C, D) are configured as outputs to send data to the 7447 decoder.
    \item The common anodes of the six seven-segment displays are connected to separate Arduino analog pins.
    \item Timer1 is configured to trigger an interrupt every 1 second to increment time.
\end{itemize}

\subsection{Displaying Time Using Multiplexing}
The clock utilizes multiplexing to drive six seven-segment displays efficiently while minimizing the number of required I/O pins.

\begin{enumerate}
    \item The digits for hours, minutes, and seconds are extracted from the stored time values using bitwise operations.
    \item The appropriate BCD values are sent to the 7447 decoder.
    \item Only one display is activated at a time using the corresponding common anode pin.
    \item A short delay ensures proper visibility before switching to the next digit.
    \item This rapid switching occurs continuously, giving the illusion that all digits are displayed simultaneously.
\end{enumerate}

\subsection{Time Keeping and Increment Logic}
The microcontroller stores time values in BCD format, ensuring efficient calculations and display updates.

\begin{itemize}
    \item The seconds value increments every time the Timer1 interrupt triggers.
    \item If seconds reach 60, they reset to 00, and the minutes value increments.
    \item If minutes reach 60, they reset to 00, and the hours value increments.
    \item If hours reach 24, they reset to 00, completing a full-day cycle.
\end{itemize}

\subsection{Timer1 Interrupt for Precise Timing}
Timer1 is configured in **Clear Timer on Compare Match (CTC) Mode** to generate precise 1-second interrupts.

\begin{enumerate}
    \item The Timer1 interrupt is triggered every second using a compare match value calculated for a 16MHz clock.
    \item When the interrupt occurs, the seconds counter increments.
    \item If a carry-over condition is met, minutes and hours are updated accordingly.
    \item The updated time values are sent to the display in the next iteration.
\end{enumerate}

\subsection{Main Loop Execution}
The `main()` function continuously updates the display while the Timer1 interrupt manages time increments in the background.

\begin{itemize}
    \item The main loop does not block execution with delays, ensuring smooth operation.
    \item Display updates occur independently of the timekeeping logic, preventing flickering or lag.
    \item The system can be expanded to include additional functionality, such as setting the time using push buttons.
\end{itemize}
