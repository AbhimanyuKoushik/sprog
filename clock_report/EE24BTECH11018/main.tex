\documentclass[12pt,a4paper]{article}
\usepackage{amsmath}
\usepackage{graphicx}
\usepackage{listings}
\usepackage{xcolor}
\usepackage{float}
\usepackage{hyperref}

\lstset{
  language=C,
  backgroundcolor=\color{black!5},
  basicstyle=\small\ttfamily,
  keywordstyle=\color{blue},
  commentstyle=\color{green!50!black},
  stringstyle=\color{red},
  breaklines=true,
  showstringspaces=false,
  numbers=left,
  numberstyle=\tiny\color{gray},
  frame=single
}

\title{Arduino Digital Clock:\\ Implementation and Analysis}
\author{Durgi Swaraj Sharma - EE24BTECH11018}
\begin{document}

\maketitle\author

\begin{abstract}
This report details the implementation of an Arduino-based digital clock that displays time in hours, minutes, and seconds format. The clock features multiplexed seven-segment displays controlled through a 7447 BCD decoder, with user inputs for time adjustment through pushbuttons. We discuss the hardware configuration, software architecture focusing on time-keeping, display multiplexing, and button debouncing techniques. The implementation emphasizes efficient port manipulation for optimal display refresh rates while maintaining minimal component count.
\end{abstract}

\section{Introduction}

The digital clock project implements a functional timekeeping device with the following key features:

\begin{itemize}
    \item 12-hour time format display (hours, minutes, seconds)
    \item Multiplexed 6-digit seven-segment display system
    \item Single 7447 BCD-to-seven-segment decoder
    \item Button controls for:
    \begin{itemize}
        \item Pausing/resuming time
        \item Adjusting hours, minutes, and seconds
    \end{itemize}
    \item Efficient display refresh through direct port manipulation
    \item Timer-based 1Hz clock with automatic time increment
\end{itemize}

The implementation employs basic embedded programming techniques including direct register manipulation, interrupt handling, and software debouncing, all optimized for the AVR microcontroller environment.

\section{Hardware Components and Circuit Design}

\subsection{Components List}

\begin{itemize}
    \item Arduino Uno R3 (ATmega328P microcontroller)
    \item Six 7-segment displays (common anode)
    \item 7447 BCD-to-seven-segment decoder
    \item Four push buttons for input
    \item Resistors (10k$\Omega$ pull-up resistors for buttons, 220$\Omega$ current limiting resistors for displays)
    \item Breadboard and connecting wires
\end{itemize}

\subsection{Circuit Connections}

The circuit employs multiplexing to minimize component count:

\subsubsection{7447 Decoder Connection}
\begin{itemize}
    \item BCD inputs (A, B, C, D) - Arduino digital pins 2-5 (PORTD 2-5)
    \item Each 7-segment display segment input (a-g) connected in parallel to the corresponding 7447 outputs
    \item Current-limiting resistors (220$\Omega$) between the 7447 outputs and display segments
\end{itemize}

\subsubsection{Display Connections}
\begin{itemize}
    \item Display common anodes connected to Arduino pins 8-13 (PORTB 0-5)
    \item The six displays represent (left to right):
    \begin{itemize}
        \item Hours tens digit (Display 0)
        \item Hours ones digit (Display 1)
        \item Minutes tens digit (Display 2)
        \item Minutes ones digit (Display 3)
        \item Seconds tens digit (Display 4)
        \item Seconds ones digit (Display 5)
    \end{itemize}
\end{itemize}

\subsubsection{Button Connections}
\begin{itemize}
    \item Four buttons connected to Arduino analog pins A0-A3 (PORTC 0-3):
    \begin{itemize}
        \item A0: Pause/resume button
        \item A1: Increment seconds button
        \item A2: Increment minutes button
        \item A3: Increment hours button
    \end{itemize}
    \item Internal pull-up resistors enabled via PORTC
\end{itemize}

\section{Software Architecture}

\subsection{Code Organization}

The software is structured into several logical components:

\begin{itemize}
    \item \textbf{Timer Initialization}: Configuration of Timer1 for 1Hz interrupts
    \item \textbf{Time Management}: Incrementation of hours, minutes, and seconds
    \item \textbf{Display Control}: Multiplexing of displays with BCD conversion
    \item \textbf{Input Handling}: Button detection and debouncing
    \item \textbf{Main Loop}: Time-to-digit conversion and display refresh
\end{itemize}

\subsection{Key Data Structures}

\begin{itemize}
    \item \textbf{Time Variables}: Three volatile variables for hours, minutes, and seconds
    \item \textbf{BCD Lookup Table}: Conversion from decimal digits to BCD values
    \item \textbf{Digit Array}: Six-element array for storing separated digits
    \item \textbf{State Variables}: Tracks clock running state and button press history
\end{itemize}

\section{Time Management Implementation}

\subsection{Timer Configuration}

The clock uses Timer1 in CTC (Clear Timer on Compare Match) mode to generate precise 1-second interrupts:

\begin{lstlisting}
void init_timer(void) {
    TCCR1A = 0;
    TCCR1B = (1 << WGM12) | (1 << CS12) | (1 << CS10);
    OCR1A = 15624;
    TIMSK1 |= (1 << OCIE1A);
}
\end{lstlisting}

The timer is configured to generate an interrupt every 1 second with a 16MHz clock frequency:
\begin{itemize}
    \item Prescaler of 1024 (CS12 | CS10)
    \item Compare value of 15624 ($\frac{16,000,000}{1024} \times 1s - 1 = 15624$)
\end{itemize}

\subsection{Time Increment Logic}

The timer interrupt service routine (ISR) increments the time values with proper rollover handling:

\begin{lstlisting}
ISR(TIMER1_COMPA_vect) {
    if(clock_running) {
        seconds++;
        if(seconds >= 60) {
            seconds = 0;
            minutes++;
            if(minutes >= 60) {
                minutes = 0;
                hours = (hours % 12) + 1;
            }
        }
    }
}
\end{lstlisting}

The time increment follows standard rollover rules:
\begin{itemize}
    \item Seconds: 0-59, rolls over to 0 and increments minutes
    \item Minutes: 0-59, rolls over to 0 and increments hours
    \item Hours: 1-12 (12-hour format), rolls over from 12 to 1
\end{itemize}

The \texttt{clock\_running} flag enables pausing of the clock when needed.

\section{Display Multiplexing System}

\subsection{BCD Conversion}

The display uses a 7447 BCD-to-seven-segment decoder, requiring BCD input:

\begin{lstlisting}
const uint8_t bcd_lookup[10] = {
    0b0000, 0b0001, 0b0010, 0b0011, 0b0100,
    0b0101, 0b0110, 0b0111, 0b1000, 0b1001
};
\end{lstlisting}

This lookup table provides the 4-bit BCD representation for each decimal digit.

\subsection{Multiplexing Implementation}

The display multiplexing function selectively activates one display at a time while setting the appropriate BCD value:

\begin{lstlisting}
void display_digit(uint8_t display, uint8_t digit) {
    // Turn off all displays
    PORTB &= ~0b00111111;
    
    // Set BCD value
    PORTD = (PORTD & 0b11000011) | ((bcd_lookup[digit] << 2) & 0b00111100);
    
    // Enable selected display
    PORTB |= (1 << display);
}
\end{lstlisting}

Key aspects of the multiplexing technique:
\begin{itemize}
    \item All displays are turned off before any change to prevent ghosting
    \item BCD data is set on PORTD pins 2-5 using masks to preserve other bits
    \item Only the desired display is activated via PORTB
\end{itemize}

This approach enables using a single 7447 decoder for all six digits, significantly reducing component count.

\section{Button Input Handling}

\subsection{Button Debouncing}

The implementation includes an effective debouncing algorithm:

\begin{lstlisting}
void check_buttons(void) {
    static uint8_t last_state = 0xFF;
    uint8_t current_state = PINC & 0x0F;
    uint8_t pressed = last_state & ~current_state;

    if(pressed) {
        // Debounce delay
        _delay_ms(50);
        current_state = PINC & 0x0F;
        pressed = last_state & ~current_state;
    }
    
    // Process button actions...
    
    last_state = current_state;
}
\end{lstlisting}

The debouncing strategy:
\begin{itemize}
    \item Tracks previous button state to detect transitions
    \item Detects button press using bitwise operations
    \item Adds a short delay and re-samples to verify the button press
    \item Only processes verified button presses
\end{itemize}

\subsection{Time Adjustment Functions}

Each button provides a specific time adjustment function:

\begin{lstlisting}
// Inside check_buttons()
if(pressed & 0x01) { // A0: Toggle clock
    clock_running ^= 1;
}
if(pressed & 0x02) { // A1: Inc seconds
    seconds = (seconds + 1) % 60;
}
if(pressed & 0x04) { // A2: Inc minutes
    minutes = (minutes + 1) % 60;
}
if(pressed & 0x08) { // A3: Inc hours
    hours = (hours % 12) + 1;
}
\end{lstlisting}

The button functions provide a simple yet complete interface for clock adjustment:
\begin{itemize}
    \item A0 toggles between running and paused states (for setting time)
    \item A1-A3 provide increment functions for each time component
    \item Each increment function handles proper range limiting
\end{itemize}

\section{Main Program Loop}

The main loop handles digit calculation and display updates:

\begin{lstlisting}
int main(void) {
    init_ports();
    init_timer();
    sei();

    uint8_t digits[6];
    
    while(1) {
        check_buttons();

        // Calculate display digits
        digits[0] = hours / 10;
        digits[1] = hours % 10;
        digits[2] = minutes / 10;
        digits[3] = minutes % 10;
        digits[4] = seconds / 10;
        digits[5] = seconds % 10;

        // Update displays
        for(uint8_t i = 0; i < 6; i++) {
            display_digit(i, digits[i]);
            _delay_us(1000);
        }
    }
    
    return 0;
}
\end{lstlisting}

This approach ensures:
\begin{itemize}
    \item Regular checking of button inputs
    \item Time-to-digit conversion for all 6 displays
    \item Rapid cycling through displays for persistence of vision
    \item Consistent refresh rate with controlled timing
\end{itemize}

\section{Optimizations}

\subsection{Port Manipulation}

The implementation uses direct port manipulation rather than Arduino functions like \texttt{digitalWrite()}:

\begin{lstlisting}
PORTB &= ~0b00111111; // Clear display enable bits
PORTB |= (1 << display); // Set specific display bit
\end{lstlisting}

Benefits of this approach:
\begin{itemize}
    \item Much faster execution (5-10× faster than \texttt{digitalWrite()})
    \item Predictable timing for multiplexing
    \item Atomic operations for display control
\end{itemize}

\subsection{BCD Lookup Table}

The BCD conversion uses a pre-calculated lookup table rather than computing BCD values on-the-fly:

\begin{lstlisting}
const uint8_t bcd_lookup[10] = {0b0000, 0b0001, ...};
\end{lstlisting}

This optimization:
\begin{itemize}
    \item Reduces computation time in the display loop
    \item Ensures constant-time BCD conversion regardless of digit value
    \item Uses program memory rather than RAM for constant data
\end{itemize}

\section{Conclusion}

The Arduino digital clock project demonstrates efficient implementation of a multiplexed display system with button-controlled time adjustment. Key achievements include:

\begin{itemize}
    \item Efficient multiplexing with a single BCD decoder for six displays
    \item Precise timekeeping using timer interrupts
    \item Effective button debouncing for reliable user input
    \item Optimized port manipulation for rapid display updates
    \item 12-hour time format with proper rollover handling
\end{itemize}

This project provides practical insights into multiplexing, emphasizing hardware-software integration, timekeeping mechanisms, and user interface design in resource-constrained environments.

\end{document}

