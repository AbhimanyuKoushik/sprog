% Main LaTeX File (main.tex)
\documentclass{article}
\usepackage{graphicx} % Required for inserting images
\usepackage{array}
\usepackage{float}
\usepackage{xcolor}
\usepackage[most]{tcolorbox}
\usepackage{circuitikz}

\title{\textbf{LAB-REPORT-1}}
\author{M. SRUJANA}

\begin{document}

\maketitle

\section{Aim :} CLOCK USING AVR.GCC WITHOUT IC'S

\section{Installation :}
\begin{enumerate}
    \item In termux: apt install avr-gcc avr-binutils avr-libc avrdude make -y
\end{enumerate}

\section{Apparatus :}
\begin{enumerate}
    \item Breadboard 
    \item 6 Seven-Segment Displays
    \item Jumper wires
    \item Arduino
    \item Arduino USB Cable 
\end{enumerate}

\section{Connections :}
Make the connections as mentioned below:
\begin{enumerate}
    \item These connections are for COMMON ANODE.
    \item Connect the common anode pin of each 7-segment display to +5V through a 220$\Omega$ resistor.
    \item Connect each segment pin (A to G) of the displays to the corresponding output pins of the Arduino.
    \item The display will be multiplexed by turning on one digit at a time while updating the corresponding segments.
\end{enumerate}

% Including Tables
 \begin{table}[h!]
    \centering
    \begin{tabular}{|c|c|}
        \hline
        \textbf{One end of Jumper Wire}  & \textbf{Another end of Jumper Wire} \\
        \hline
          Digital pin 0 & Push button 16 \\
          Digital pin 1 & Push button 17 \\
          Digital pin 2 & LCD pin 4 \\
          Digital pin 3 & LCD pin 6 \\
          Digital pin 4 & LCD pin 11 \\
          Digital pin 5 & LCD pin 12 \\
          Digital pin 6 & LCD pin 13 \\
          Digital pin 7 & LCD pin 14 \\
          Digital pin 8 & Push button 18 \\
          Digital pin 9 & Push button 19 \\
          Digital pin 10 & Push button 20 \\
          Digital pin 11 & Push button 21 \\
          Digital pin 12 & Push button 22 \\
          Digital pin 13 & Push button 23 \\
          Analog pin A1 & Push button 15 \\
          Analog pin A2 & Push button 14 \\
          Analog pin A3 & Push button 13 \\
          Analog pin A4 & Push button 12 \\
          Analog pin A5 & Push button 11 \\
          Analog pin A0 & Push buttons 1-10 (digit buttons)\\
          LCD pin 1 & Ground \\
          LCD pin 2 & 5V \\
          LCD pin 15 & 5V via 1k $\Omega$ resistor \\
          LCD pin 16 & Ground \\
          LCD pin 3 & Ground via 1.5 k $\Omega$ resistor \\
          LCD pin 5 & Ground \\
          LCD pin 5 & All push buttons \\
          
        \hline
    \end{tabular}
\end{table}

 \begin{table}[h!]
    \centering
    \begin{tabular}{|c|c|}
        \hline
        \textbf{Button number}  & \textbf{Function} \\
        \hline
           1 - 10 & Digits 0 - 9 \\
           11 & Clear \\
           12 & $\ln{(x)}$ and $\log{(x)}$ \\
           13 & Right Parenthesis \\
           14 & $\sin{(x)}$, $\cos{(x)}$, and $\tan{(x)}$ \\
           15 & $e$ and $\pi$ \\
           16 & Backspace \\
           17 & Decimal Point \\
           18 & Equal To \\
           19 & Left Parenthesis \\
           20 & Division (/)\\
           21 & Multiplications (*)\\
           22 & Subtraction(-) \\
           23 & Addition (+) \\
        \hline
    \end{tabular}
\end{table}


% Including Figures

\section{Code : }
\begin{tcolorbox}[colback=green!10, colframe=green!50!black, breakable]
\begin{verbatim}
#define F_CPU 16000000UL
#include <avr/io.h>
#include <avr/interrupt.h>
#include <util/delay.h>

const uint8_t digit_map[] = {
    0b00000000, 0b11100100, 0b10010000, 0b11000000, 0b01100100,
    0b01001000, 0b00001000, 0b11100000, 0b00000000, 0b01000000
};

volatile uint8_t hours = 1, minutes = 11, seconds = 30;
uint8_t digits[6];

void update_digits() {
    digits[0] = hours / 10;
    digits[1] = hours % 10;
    digits[2] = minutes / 10;
    digits[3] = minutes % 10;
    digits[4] = seconds / 10;
    digits[5] = seconds % 10;
}

void update_time() {
    seconds++;
    if (seconds >= 60) { seconds = 0; minutes++; }
    if (minutes >= 60) { minutes = 0; hours++; }
    if (hours >= 24) { hours = 0; }
    update_digits();
}

ISR(TIMER1_COMPA_vect) {
    update_time();
}

void display_digit(uint8_t display, uint8_t digit) {
    PORTB &= ~(0b00011110);
    PORTC &= ~(0b00000011);
    PORTD = digit_map[digit];
    if (digit == 0 || digit == 1 || digit == 7) { PORTB |= (1 << PB0); }
    else { PORTB &= ~(1 << PB0); }
    if (display < 4) { PORTB |= (1 << (display + 1)); }
    else { PORTC |= (1 << (display - 4)); }
    _delay_ms(2);
}

int main(void) {
    DDRD |= 0b11111100;
    DDRB |= (1 << PB0);
    DDRB |= (1 << PB1) | (1 << PB2) | (1 << PB3) | (1 << PB4);
    DDRC |= (1 << PC0) | (1 << PC1);
    update_digits();
    TCCR1B |= (1 << WGM12) | (1 << CS12) | (1 << CS10);
    OCR1A = 15625;
    TIMSK1 |= (1 << OCIE1A);
    sei();
    while (1) {
        for (uint8_t i = 0; i < 6; i++) {
            display_digit(i, digits[i]);
        }
    }
}
\end{verbatim}
\end{tcolorbox}

\section{Execution : }
\begin{enumerate}
    \item To compile main.c file : avr-gcc -mmcu=atmega328p -Os -o main.elf main.c
    \item To convert main.elf file to main.hex : avr-objcopy -O ihex -R .eeprom main.elf main.hex
    \item Moving the Compiled File to ArduinoDroid : mv main.hex /sdcard/ArduinoDroid/precompiled
\end{enumerate}

\end{document}

